\documentclass[10pt,letterpaper]{article}
\usepackage[top=0.85in,left=2.75in,footskip=0.75in]{geometry}

% amsmath and amssymb packages, useful for mathematical formulas and symbols
\usepackage{amsmath,amssymb}

% Use adjustwidth environment to exceed column width (see example table in text)
\usepackage{changepage}

% Use Unicode characters when possible
\usepackage[utf8x]{inputenc}

% textcomp package and marvosym package for additional characters
\usepackage{textcomp,marvosym}

% cite package, to clean up citations in the main text. Do not remove.
\usepackage{cite}

% Use nameref to cite supporting information files (see Supporting Information section for more info)
\usepackage{nameref,hyperref}

% line numbers
\usepackage[right]{lineno}

% ligatures disabled
\usepackage{microtype}
\DisableLigatures[f]{encoding = *, family = * }

% color can be used to apply background shading to table cells only
\usepackage[table]{xcolor}

% array package and thick rules for tables
\usepackage{array}

% create "+" rule type for thick vertical lines
\newcolumntype{+}{!{\vrule width 2pt}}

% create \thickcline for thick horizontal lines of variable length
\newlength\savedwidth
\newcommand\thickcline[1]{%
  \noalign{\global\savedwidth\arrayrulewidth\global\arrayrulewidth 2pt}%
  \cline{#1}%
  \noalign{\vskip\arrayrulewidth}%
  \noalign{\global\arrayrulewidth\savedwidth}%
}

% \thickhline command for thick horizontal lines that span the table
\newcommand\thickhline{\noalign{\global\savedwidth\arrayrulewidth\global\arrayrulewidth 2pt}%
\hline
\noalign{\global\arrayrulewidth\savedwidth}}


% Remove comment for double spacing
%\usepackage{setspace} 
%\doublespacing

% Text layout
\raggedright
\setlength{\parindent}{0.5cm}
\textwidth 5.25in 
\textheight 8.75in

% Bold the 'Figure #' in the caption and separate it from the title/caption with a period
% Captions will be left justified
\usepackage[aboveskip=1pt,labelfont=bf,labelsep=period,justification=raggedright,singlelinecheck=off]{caption}
\renewcommand{\figurename}{Fig}

% Use the PLoS provided BiBTeX style
\bibliographystyle{plos2015}

% Remove brackets from numbering in List of References
\makeatletter
\renewcommand{\@biblabel}[1]{\quad#1.}
\makeatother

% Leave date blank
\date{}


% Header and Footer with logo
\usepackage{lastpage,fancyhdr,graphicx}
\usepackage{epstopdf}
\pagestyle{myheadings}
\pagestyle{fancy}
\fancyhf{}
\setlength{\headheight}{27.023pt}
\lhead{\includegraphics[width=2.0in]{PLOS-submission.eps}}
\rfoot{\thepage/\pageref{LastPage}}
\renewcommand{\footrule}{\hrule height 2pt \vspace{2mm}}
\fancyheadoffset[L]{2.25in}
\fancyfootoffset[L]{2.25in}
\lfoot{\sf PLOS}

%% Include all macros below

\newcommand{\lorem}{{\bf LOREM}}
\newcommand{\ipsum}{{\bf IPSUM}}

%% END MACROS SECTION
\usepackage[backend=bibtex]{biblatex}
\addbibresource{mybib.bib}


\usepackage{Sweave}
\begin{document}
\Sconcordance{concordance:LaRoche_Manuscript_PLOS.tex:LaRoche_Manuscript_PLOS.Rnw:%
1 127 1 49 0 1 26 1 1 1 16 1 1 1 39 59 1 1 21 2 1 1 7 26 1 1 30 1 %
1 1 12 1 1 1 90 1 1 1 16 1 1 1 15 1 1 1 20 1 1 1 9 1 1 1 13 1 1 1 %
17 1 1 1 7 1 1 1 23 1 1 1 22 1 1 1 24 4 1 1 3 25 1 1 27 1 1 1 3 20 %
1 1 11 1 1 1 13 2 1 1 7 1 1 1 6 2 1 1 5 1 1 1 21 1 1 1 21 1 1 1 4 %
1 1 1 17 6 1 1 3 29 1 1 27 2 1 1 3 16 1}

\vspace*{0.2in}
% Title must be 250 characters or less.
\begin{flushleft}
{\Large
\textbf\newline{A method for identifying households at high risk for mosquito borne illnesses.} % Please use "title case" (capitalize all terms in the title except conjunctions, prepositions, and articles).
}
\newline
% Insert author names, affiliations and corresponding author email (do not include titles, positions, or degrees).
\\
Dominic D. LaRoche\textsuperscript{1*},
Melanie L. Bell\textsuperscript{1},
Kacey C. Ernst\textsuperscript{1},
\\
\bigskip
\textbf{1} Mel & Enid Zuckerman College of Public Health, University of Arizona , Tucson, AZ, USA
\\
\bigskip

% Use the asterisk to denote corresponding authorship and provide email address in note below.
* dlaroche@email.arizona.edu

\end{flushleft}



% Please keep the abstract below 300 words
\section*{Abstract}
Mosquito borne illnesses are a significant threat to public health in countries where these diseases are either endemic or epidemic. Concerted efforts have been made in the past decade to reduce and in some cases eliminate mosquito borne diseases with the use of prophylactic interventions. The World Health Organization recommends preferential administration of interventions to those with the highest disease burden. However,  previous research has also identified the benefit of additionally targeting interventions at those with the highest risk of infections. We develop a methodology for identifying the highest risk households on landscape by combining both health risk and infection risk.  We use census information to determine health risk and we use a topographic wetness index to determine infection risk.  We implement the method to evaluate the administration of interventions at two sites in Kenya. We find preferential administration of interventions at the high-elevation epidemic-prone site but not at the low-elevation endemic site. Our results have important implications for assessing the administration of interventions in the battle against mosquito borne illnesses.

\linenumbers

\section*{Introduction}
%introduce the problem
Malaria and other mosquito borne illnesses are considered a significant threat to public health and a socio-economic burden in countries where these diseases are either endemic or epidemic \cite{Crouch}. Concerted efforts have been made in the past decade to reduce and in some cases eliminate malaria specifically. Many national strategic plans to reduce or eliminate malaria are in their third generation.  The World Health Organization recommends prioritizing the administration of interventions to pregnant women and young children followed by progressively achieving intervention coverage of all community members. The preferential administration of interventions to pregnant women and young children reflects the disproportionate disease burden borne by this group \cite{Bousema2012}. However,  previous research has identified the benefit of additionally targeting interventions at those with the highest risk of infections \cite{Schantz-Dunn2009}. Identifying individuals which are both vulnerable to infection and likely to be exposed to an infected mosquito is therefore a priority.  However, identifying high risk individuals can be costly and inefficient.\\

The Topographic Wetness Index (TWI) \cite{Beven1979} derived from freely available remotely-sensed topographic data has been previously investigated as a tool for assessing risk of malaria infection \cite{Cohen2008,Cohen2010}. TWI can potentially identify areas where water is likely to pool and Anopheles densities are likely to be higher. This method, therefore, has the potential to both inform new distribution campaigns and evaluate the efficacy of existing campaigns.  However, traditional TWI algorithms are complicated and generally require specialized software to implement.  Moreover, simply identifying areas where are mosquitoes are likely to breed does not account for the differential health risk of the exposed population.  We develop a simple, matrix-based, methodology using topographic data and combine this with a household census of demographics to identify high risk households.  We apply the method to two sites in Kenya where malaria is either endemic or epidemic-prone.  We compare our methodology to a traditional Topographic  Wetness Index by using knowledge of intervention use for two interventions at two sites in Kenya. \\%This method, therefore, has the potential to both inform new distribution campaigns and evaluate the efficacy of existing campaigns.



\section*{Materials and Methods}%main idea is to use traditional or aspect variance methods and see if there is a substantial difference.  Would be best 
Every household on a landscape will vary with respect to both the risk of exposure to mosquitoes and the number of at risk individuals in the household. We develop a method to combine these two risk factors into an overall risk score for each household within a management area in order to identify the households which will yield the greatest benefit from the application of limited resources.\\

\subsection{Topographical Wetness Index}

The Topographical Wetness index was originally introduced by Beven and Kirby in 1979 (\cite{Beven1979}).  TWI combines a measure of the amount of upstream drainage area with the local slope to determine the amount of wetness likely to accumulate at a point and is defined by Beven and Kirby generally as:
$$ln\frac{a}{tanb},$$
where $a$ is the local up-slope area (local area draining to the point) and $tanb$ is the local slope in radians.  The TWI is designed to predict the amount of water that is likely to flow into an area, based on surface topology, and the rate at which this water will flow out of an area.  Since TWI was first defined numerous methods have been developed to calculate it and several review papers have been published \cite{Quinn1995,Sørensen2006} as well as alternative modeling strategies developed \cite{Grabs2009}.  Areas with high in-flow and low out-flow are likely provide habitat for breeding mosquitoes. \\

The primary goal of using TWI in this application is to identify areas where mosquitoes are likely to breed.  However, the general TWI was originally designed to model surface water flow and not necessarily to identify areas where water will pool. A simpler algorithm for identifying only areas where water is likely to pool may perform as well, or better, than a general TWI algorithm and would not require specialized GIS software for implementation.  To examine this, we implement 2 different algorithms which differ in their calculation of the up-slope area and local slope: 1) the SAGA Wetness Index \cite{Bohner2002} (hereafter "general TWI"), and 2) a simplified topographic wetness index which simply identifies depressions without regard to up-slope area (hereafter "restricted TWI").  \\


We carried out all analyses using the statistical programming language R version 3.2.3 \cite{RCoreTeam2015}, with the exception of the SAGA TWI which was calculated with the SAGA open-source GIS software \cite{Bohner2006}.  Details for calculation of the general TWI are available in the software documentation available at \href{http://www.saga-gis.org/saga_module_doc/2.1.4/ta_hydrology.html}{ http://www.saga-gis.org\cite{Conrad2015}.  The method involves, for every point on the landscape:  calculating the up-slope area from which water will flow to the point,  calculating the catchment area, and the slope of the catchment area.  Areas with a larger up-slope area and a smaller catchment area slope are predicted to have more water accumulation.\\

In contrast to the relatively complicated calculation of a general TWI, we suggest a simple model based on identifying depressions with low outflow and disregarding the up-slope area.  We first identify depressions and valleys by identifying pixels which are lower than the average of their neighbors.  We do this by first calculating the average elevation of the surrounding pixels for each pixel ($p$) in the landscape at three resolutions of increasing size:  

$$\mu_{i,j} = \frac{\sum_{p \in j}e_p}{|p \in j|},$$

where $\mu_{i,j}$ is the average elevation of the pixels in the window of size $j$ surrounding pixel $i$.  The size of the windows is somewhat arbitrary but the idea is to identify small depressions from large ones so we suggest $90m \times 90m$, $210m \times 210m$, and $330m \times 330m$.  These sizes may be adjusted or tuned to a particular landscape if information is available to do so.  We then subtract the mean elevation from the actual elevation at each pixel from the three calculated averages and sum these differences.\\

$$\delta e_{i,j} = e_i - \mu_{i,j} $$
$$\Delta e_{i} = \sum_{j \in 1,2,3} \delta e_{i,j}$$

We then identify areas with low outflow by calculating the variance of the aspects (direction the slope faces) of neighboring pixels, $\theta$.  Since aspect is measured on a circular scale from 1 to 360 the variance of two aspects which are in fact quite close to each other but on opposite sides of the circular scale reset at 360 would be artificially high.  For example, the aspect of two points facing close to due north could be 359 and 1.  The variance of these two points would be $var(359,1)= 64082$.  However, the variance of two aspects of equal distance but not on opposite sides of the circular reset would be $var(1,3) = 2$.  In order to mitigate this difficulty we first translate each aspect to a cardinal direction denoted 1, 2, 3, or 4. We then calculate the variance of the aspects of the cells in each of 3 moving windows the same size as above ($90m \times 90m$, $210m \times 210m$, and $330m \times 330m$) and sum the resulting variances for each pixel:

$$\theta_{i} = \sum_{j \in 1,2,3} var(a_{i \in j}),$$

where $a_{i \in j}$ are the set of aspects in the moving window $j$  around pixel $i$.  The final wetness index is then calculated as, 
$$w_i = -1 \times \Delta e_i \times \theta_i.$$

Relatively large positive values of $w_i$ are expected to have higher wetness than small positive numbers.  Negative numbers indicate a ridge or peak and are therefore expected to be dry.\\
 

We assign each household a risk for exposure to mosquitoes (exposure risk hereafter) by deriving a continuous risk surface over the study area from each TWI algorithm.  We assume the mosquito exposure risk of a household is inversely related to the distance to one or more of these high-wetness areas.  Therefore,  we apply a Gaussian filter with $\sigma = 10$ to create a weighted average of mosquito risk for each cell in the study area.  The Gaussian filter is an isometric 2-dimensional smoothing function with a Gaussian kernel.  The value of $\sigma$ determines the degree of smoothing and should be scaled appropriately to match the resolution of the data.  The value of $\sigma$ can be tuned for a specific application if data is available and appropriate to do so. Each house is then assigned the risk score of the cell it is in or, for high resolution topographical data such as LiDAR (laser radar), the average of the cells a property occupies.\\


\subsection{Individual Health Risk}
Each household varied with respect to both the risk of exposure to mosquitoes and the number of at-risk individuals in the household. The household risk formulas will vary depending on the disease under study and the most vulnerable population(s) for that disease.  We recommend a simple additive risk score, like the score formula provided below developed for malaria, based on expert opinion or relevant literature. The sole purpose of the health risk score is to differentiate households with high risk from those with low risk.  More complicated formulas can be constructed but we believe a simple formula will be easier to interpret and adequate for most applications.\\

\subsection{Overall Risk}

To be at risk for a poor outcome a person must 1) come in contact with a disease harboring mosquito,  and 2) be inherently vulnerable to infection (e.g. very young or very old).  We create two risk scores representing each of these types of risk for every household on the landscape.  Since these risks will be calculated on different scales we center at 0 and standardize risk scores so that they are scale-independent.  These two scores can then be combined with a weighted sum to create an overall risk score for the household.  Weights in the sum are determined by expert opinion, in our example below we use equal weights.  This method lends itself well to the addition of additional risk scores, such as the distance to a health care provider, or even risk attenuating factors such as the presence of window screens, which can all be summed together with appropriate weights. 


